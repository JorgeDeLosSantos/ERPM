\chapter{Interfaces gráficas de usuario}

Comunmente se denomina interfaz gráfica de usuario (GUI) a un conjunto de objetos gráficos mostrados en una 
ventana, que el usuario de un ordenador utiliza para interactuar con el sistema operativo o el software en 
ejecución. En MATLAB las GUI normalmente incluyen botones, campos de texto, check boxes, menús desplegables, 
list boxes, tablas, entre otros que facilitan la interacción del usuario. Cada uno de estos elementos gráficos 
es programado para responder a una determinada acción realizada por el usuario.

\section{El hola mundo en GUI}

\textit{Desarrolle una interfaz gráfica que contenga un botón, el cual al ser presionado por el usuario deberá \textit{responder} mostrando un cuadro de dialogo (msgbox) con la clásica cadena \textit{hola mundo}.}



\begin{verbatim}
function HolaMundo
figure('MenuBar','None',...
    'NumberTitle','off',...
    'Name','Hola Mundo');
 
uicontrol('style','push',...
    'String','Botón',...
    'Callback','msgbox(''Hola mundo'')');
end
\end{verbatim}

\section{Ventana multicolor}

\textit{Programe una interfaz gráfica que no contenga control gráfico alguno, y que solamente cambie de color a cada cierto tiempo, esto hasta que el usuario cierre la ventana correspondiente.}



\begin{verbatim}
function CambiaColor
f = figure('MenuBar','none',...
    'NumberTitle','off',...
    'Name','Cambia Color');
 
while ishandle(f)
    set(f,'Color',rand(1,3));
    pause(0.5);
    drawnow;
end
 
end
\end{verbatim}

\section{Lista de archivos M.}

\textit{Desarrolle una GUI con un List Box, el cual debe contener una lista de los nombres de archivos *.m ubicados en el mismo directorio. Al presionar cada uno de los elementos de la lista debe abrir el archivo seleccionado en el editor de MATLAB.}



\begin{verbatim}
function ListaArchivosM
figure('MenuBar','none',...
    'NumberTitle','off',...
    'Name','Lista Archivos',...
    'Position',[0 0 200 300]);
centerfig();
 
archivos_m = dir('*.m');
archivos_m = struct2cell(archivos_m);
uicontrol('style','listbox',...
    'String',archivos_m(1,:),...
    'Units','Normalized',...
    'Position',[0.02 0.02 0.96 0.96],...
    'Callback',@edit_call);
 
    function edit_call(src,~)
        str = get(src,'String');
        k = get(src,'Value');
        edit(str{k});
    end
end
\end{verbatim}

\section{Mini Calculadora}

\textit{Diseñe y desarrolle una interfaz gráfica de usuario que asemeje el comportamiento de una calculadora muy sencilla. Debe contener cuatro botones correspondientes a los operadores aritméticos básicos, dos campos editables que permitan insertar los datos de entrada y otro campo estático o editable que permita mostrar el resultado de la operación realizada.}



\begin{verbatim}
function MiniCalculadora
figure('MenuBar','None',...
    'NumberTitle','off',...
    'Name','Mini Calculadora',...
    'Resize','off',...
    'Position',[0 0 300 150]);
centerfig();
 
% ============================= DATOS ===============================
panel_datos = uipanel('Units','Pixels',...
    'Position',[10 50 280 95]);
 
uicontrol(panel_datos,'Style','text',...
    'String','# 1',...
    'Units','Normalized',...
    'Position',[0 0.67 0.4 0.25]);
hN1=uicontrol(panel_datos,'Style','edit',...
    'String','',...
    'Units','Normalized',...
    'Position',[0.45 0.72 0.5 0.25]);
 
uicontrol(panel_datos,'Style','text',...
    'String','# 2',...
    'Units','Normalized',...
    'Position',[0 0.33 0.4 0.25]);
hN2=uicontrol(panel_datos,'Style','edit',...
    'String','',...
    'Units','Normalized',...
    'Position',[0.45 0.38 0.5 0.25]);
 
uicontrol(panel_datos,'Style','text',...
    'String','Resultado',...
    'Units','Normalized',...
    'Position',[0 0 0.4 0.25]);
hR=uicontrol(panel_datos,'Style','edit',...
    'String','',...
    'Units','Normalized',...
    'Position',[0.45 0.05 0.5 0.25],...
    'BackG',ones(1,3)*0.8);
 
% ===================== BOTONES OPERADORES ============================
panel_botones = uipanel('Units','Pixels',...
    'Position',[10 5 280 40]);
 
OPERADORES = '+-*/';
 
for k = 1:length(OPERADORES)
    uicontrol(panel_botones,'style','push',...
        'String',OPERADORES(k),...
        'Units','normalized',...
        'Position',[(k-1)*(1/4) 0 1/4 1],...
        'FontSize',16,...
        'FontWeight','bold',...
        'Callback',@calcular);
end
 
    function calcular(src,~)
        n1 = get(hN1,'String'); % Primer número
        n2 = get(hN2,'String'); % Segundo número
        oper = get(src,'String'); % Operador
        set(hR,'String',num2str(eval([n1,oper,n2])));
    end
 
end
\end{verbatim}

\section{Visor de imágenes (Nivel I)}

\textit{Desarrolle una GUI que funcione como un visor de imágenes simple, el cual debe contener un menú Archivo y dentro de este dos sub-menús llamados Abrir y Salir. La opción Abrir debe mostrar al usuario un explorador de archivos interactivo (uigetfile) que le permita seleccionar una imagen en formato PNG y enseguida mostrarla en un axes ubicado dentro la misma GUI (utilice las funciones imread e imshow para la manipulación de la imagen). La opción Salir, en este caso, es auto-descriptiva.}



\begin{verbatim}
function AbrirImagen
f = figure('MenuBar','None',...
    'NumberTitle','off',...
    'Name','Abrir imagen');

% ====================== MENÚ ============================
hMenu = uimenu(f,'Label','Archivo');
uimenu(hMenu,'Label','Abrir','Callback',@abrir_img);
uimenu(hMenu,'Label','Salir','Callback','close(gcf)');

% ===================== AXES ============================
ax = axes('Units','Normalized',...
    'Position',[0 0 1 1],...
    'Visible','off');

    function abrir_img(~,~)
        [filename, pathname] = uigetfile('*.PNG', 'Seleccione una imagen');
        if isequal(filename,0) || isequal(pathname,0)
            return;
        else
            X = imread(fullfile(pathname,filename));
            imshow(X);
        end
    end
end
\end{verbatim}
